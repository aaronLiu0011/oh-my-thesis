\documentclass{article}
\usepackage{graphicx} % Required for inserting images
\usepackage{amsmath}
\usepackage{amssymb}
\usepackage{hyperref}


% =========================================================

\begin{document}

\newpage
\tableofcontents
\newpage

\listoffigures
\newpage

\listoftables
\newpage

\pagenumbering{arabic}


\section{Introduction}

\subsection{Research Question}

This study examines whether the abortion restrictions enacted in the aftermath of the
\textit{Dobbs v.~Jackson Women’s Health Organization} ruling produced measurable effects on
sexually transmitted infection (STI) outcomes in the United States. The \textit{Dobbs} decision
fundamentally altered the institutional environment surrounding reproductive health by removing
federal constitutional protection for abortion and returning regulatory authority to individual
states. Several states responded by enforcing restrictive laws almost immediately, while others
maintained the status quo. This abrupt divergence provides a policy intervention that is plausibly
exogenous with respect to underlying STI trends.

The central question is whether these restrictive abortion policies contributed to changes in the
incidence of major reportable STIs—including syphilis, gonorrhea, and chlamydia—during the two
years following the ruling. STIs have exhibited sustained growth in the United States over the
past decade, and recent policy debates have highlighted the importance of reproductive-health
infrastructure and sexual health services in shaping disease transmission. Despite these broader
concerns, empirical evidence linking abortion policy to STI trends remains limited, and the
potential spillovers of reproductive policy into other dimensions of public health are still not
well understood.

This study seeks to provide a systematic assessment of whether the policy shift in 2022 induced
observable departures from the counterfactual STI trajectories that states would have followed in
the absence of the ruling. The analysis is grounded in state-year panel data spanning more than a
decade before and after the decision, allowing for the estimation of both average treatment
effects and dynamic changes over time. By bringing together policy variation, detailed public
health data, and modern causal inference methods, the study aims to clarify whether the post-
\textit{Dobbs} abortion restrictions have had unintended consequences for sexual health outcomes.


\subsection{Motivation}

Reproductive-health policies often influence a wide range of behavioural and institutional
responses beyond fertility outcomes. Following the \textit{Dobbs} decision, substantial public
debate has focused on access to reproductive care, the reconfiguration of healthcare systems, and
potential spillovers to related health behaviours. At the same time, STI rates in the United
States have exhibited persistent growth in the past decade, with notable heterogeneity across
states. Despite these developments, empirical evidence on the causal link between post-\textit{Dobbs}
abortion restrictions and STI outcomes remains limited. The motivation of this study is to fill
this gap by providing credible causal estimates based on long-term state-level panel data.

\subsection{Contribution}

This study contributes to the literature in several ways. First, it provides new causal evidence on 
how abortion restrictions influence sexual-health outcomes, an area that has received limited 
empirical attention. While existing work has examined the effects of reproductive-health policy on 
fertility, abortion incidence, and related services, the potential spillovers onto STI outcomes have 
not been systematically analysed. Second, the study constructs a state-year panel from 2010 to 2023 
that combines surveillance data, demographic indicators, and socioeconomic controls. This dataset 
supports both cross-state comparisons and dynamic analysis over an extended period. Third, the 
empirical design integrates Difference-in-Differences estimation, event-study specifications, and 
synthetic control methods. This combination provides a transparent framework for assessing the 
robustness of the findings and evaluating the credibility of identification. By triangulating 
evidence across complementary approaches, the analysis aims to produce a coherent assessment of the 
policy’s impact.

\subsection{Summary of Findings}

The results indicate that states implementing abortion restrictions after the \textit{Dobbs} ruling 
experienced increases in STI indices relative to states that maintained access. Event-study 
estimates reveal stable pre-treatment patterns and a clear divergence beginning in 2022, consistent 
with the timing of the policy change. Synthetic control analyses for selected states display similar 
post-treatment separations from their synthetic counterparts, and placebo exercises do not show 
comparable effects in periods without policy changes or among states that did not adopt 
restrictions. The convergence of these findings across multiple identification strategies suggests 
that the observed increase in STI outcomes reflects the causal impact of abortion restrictions 
rather than artefacts of model specification or idiosyncratic data patterns.


\subsection{Roadmap}

The remainder of the thesis proceeds as follows.
Section~\ref{sec:institution} describes the institutional and policy background relevant to the
\textit{Dobbs} ruling and the variation in state-level abortion restrictions.
Section~\ref{sec:data} presents the data sources, variable construction, and descriptive
statistics.
Section~\ref{sec:strategy} outlines the empirical strategy, including the baseline Difference-in-
Differences model, event-study framework, and robustness checks.
Section~\ref{sec:results} reports the main results.
Section~\ref{sec:discussion} discusses the broader interpretation of the findings and situates
them within the existing literature.
Section~\ref{sec:conclusion} concludes.

% =========================================================

\section{Institutional Background}\label{sec:institution}
11111
\subsection{Abortion Policy After \textit{Dobbs}}
\subsection{Trigger Laws and State-Level Variation}
\subsection{STI Trends and Public Health Framework}

% =========================================================
\section{Data}\label{sec:data}

\subsection{Data Sources}

1111
\subsection{Outcome Variables}
\subsection{Treatment Variable}
\subsection{Socioeconomic and Demographic Controls}
\subsection{Sample Construction and Cleaning}
\subsection{Descriptive Statistics}

% =========================================================

\section{Empirical Strategy}\label{sec:strategy}

This section outlines the empirical approach used to estimate the impact of post-\textit{Dobbs}
abortion restrictions on sexually transmitted infection outcomes. The design takes advantage of
a sharp institutional shift produced by the 2022 Supreme Court decision, which removed federal
constitutional protection for abortion and led to immediate policy changes in a subset of states.
The staggered response across states creates a treated–control structure in which some states
implemented restrictive laws while others maintained existing access. This variation, combined
with a long state-year panel from 2010 to 2023, provides a suitable environment for identifying
causal effects using Difference-in-Differences methods, dynamic event-study estimators, and
synthetic control analysis. These approaches enable the estimation of both average and evolving
policy effects while offering multiple avenues for assessing identification and robustness.

\subsection{Baseline Difference-in-Differences Model}

The analysis begins with a two-way fixed-effects Difference-in-Differences specification. Let
$Y_{st}$ denote the STI outcome of interest for state $s$ in year $t$. The treatment indicator
$\text{Policy}_{st}$ equals one if a state enforces a restrictive abortion statute in that year
and zero otherwise. The baseline model is given by

\begin{equation}
    Y_{st} = \beta\,\text{Policy}_{st} + X_{st}'\Gamma + \alpha_s + \lambda_t + \varepsilon_{st}.
\end{equation}

The vector $X_{st}$ consists of time-varying covariates that capture demographic and
socioeconomic conditions that may influence STI dynamics and correlate with policy adoption.
State fixed effects $\alpha_s$ remove persistent differences across states, such as long-standing
cultural, institutional, or healthcare infrastructure characteristics. Year fixed effects
$\lambda_t$ absorb national developments that influence all states simultaneously, including
changes in reporting practices or nationwide health shocks. Standard errors are clustered at the
state level to account for serial correlation.

Under the identifying assumptions discussed in the next subsection, the coefficient $\beta$
captures the average treatment effect of abortion restrictions on STI outcomes. This estimate
reflects how the treated states’ trajectories diverged from their counterfactual paths represented
by control states. The specification serves as the foundation for subsequent analyses and provides
a benchmark against which more flexible approaches can be compared.

\subsection{Identification Assumptions}

The interpretation of the Difference-in-Differences coefficient as a causal effect depends on the
plausibility of several assumptions. A first requirement is that treated and untreated states would
have followed parallel trends in STI outcomes in the absence of policy changes. While the
counterfactual cannot be observed, the event-study results presented below allow an empirical
assessment by examining pre-treatment dynamics. Before the policy shift, STI trends across states
were relatively stable, and there is no indication of systematic divergence between states that
would later adopt restrictive laws and those that would not.

A second requirement concerns the absence of anticipatory behaviour. Individuals and institutions
should not alter their behaviour before the formal implementation of abortion restrictions. The
leak of the draft \textit{Dobbs} opinion raises the possibility of early adjustments, particularly
in the months immediately preceding the official ruling. In the empirical analysis, the estimates
for periods just before the policy change allow for an assessment of whether such anticipatory
behaviour is likely to affect the estimates.

A third assumption involves the absence of spillovers across states. Policy changes in one state
should not affect outcomes in another. In practice, some level of spillover may occur due to
cross-state mobility for reproductive care or differences in STI reporting practices. Such
spillovers, if present, are likely to attenuate differences by reducing the contrast between
treated and control states, implying that the estimated effect can be interpreted as a lower bound
of the true policy impact.

Together, these assumptions provide the conceptual basis for the identification strategy. Their
empirical plausibility is strengthened by the event-study results, which show stable pre-treatment
patterns and a clear separation emerging only after the policy shock.

\subsection{Event Study Specification}\label{sec:eventstudy}

To assess pre-treatment dynamics and document the evolution of treatment effects over time, the
DID model is expanded into an event-study specification. The model replaces the single treatment
indicator with a set of relative-time indicators that capture periods before and after the policy
change:

\begin{equation}
    Y_{st} = \alpha_s + \lambda_t +
    \sum_{k=-M}^{K} \beta_k D_{st}^{(k)} + X_{st}'\Gamma + \varepsilon_{st},
\end{equation}

where $D_{st}^{(k)}$ equals one if year $t$ is $k$ years relative to the implementation year,
with $k=0$ corresponding to 2022. The omitted category is $k=-1$, allowing the remaining
coefficients to be interpreted relative to the year immediately preceding the policy shift.

The event-study serves two purposes. First, the coefficients associated with pre-treatment periods
provide evidence on the validity of the parallel-trends and no-anticipation assumptions. A lack of
significant movement in these coefficients strengthens the credibility of the causal
interpretation. Second, the post-treatment coefficients trace the dynamic path of the policy’s
effects, distinguishing immediate responses from more gradual adjustments. The timing of the
\textit{Dobbs} ruling, concentrated in a single year with a substantial never-treated group,
circumvents common concerns regarding staggered adoption and heterogeneous treatment timing,
making the event-study estimator well suited for this setting.

\subsection{Synthetic Control Design}

To complement the DID and event-study results, synthetic control methods are employed for selected
treated states. For each treated state examined, a synthetic counterpart is constructed as a
convex combination of untreated states chosen to closely reproduce the treated state’s
pre-treatment STI trajectory. The weights are selected to minimize discrepancies in outcomes and
predictor variables during the pre-2022 period.

A credible synthetic control requires a strong pre-treatment fit. The root mean squared prediction
error (RMSPE) is used to assess fit quality, with a lower RMSPE indicating a more reliable
counterfactual. Following the construction of the synthetic unit, the analysis compares the paths
of the treated state and its synthetic counterpart after 2022. A widening post-treatment gap is
interpreted as the effect of the policy. To evaluate statistical significance, placebo procedures
are implemented by repeating the synthetic control construction for each state in the donor pool.
The distribution of these placebo gaps provides a benchmark for assessing the magnitude of the
observed effects.

\subsection{Robustness Checks}

Several robustness checks assess whether the findings depend on particular modelling choices or
data features. A first exercise assigns the treatment to a pre-\textit{Dobbs} period in which no
policy change occurred. If the empirical strategy were sensitive to unrelated fluctuations, this
placebo treatment would generate effects similar to the baseline estimates. A lack of significant
placebo effects therefore reinforces the credibility of the main results.

A second robustness exercise arises within the synthetic control framework. By applying the same
construction to each state in the donor pool, the analysis generates a distribution of placebo
paths for untreated states. Comparing the magnitude of the actual post-treatment gap to this
distribution provides a non-parametric test of the policy’s impact.

A third robustness exercise addresses the exceptional nature of the COVID-19 period. The years
2020 and 2021 were marked by disruptions in healthcare utilisation, reporting, and behavioural
patterns relevant to STI transmission. Including these years as part of the pre-treatment period
could distort the counterfactual trend. The main specification therefore omits these years from
the estimation sample, and additional analyses confirm that the key findings are not driven by the
treatment of the pandemic period.

Across all robustness exercises, the estimated effects remain consistent in sign and magnitude,
providing further support for the interpretation that the post-\textit{Dobbs} increase in STI
outcomes reflects the causal impact of abortion restrictions rather than artefacts of model
specification or data irregularities.

\subsection{Heterogeneity Analysis}

To examine whether the policy effect varies across states, the baseline model is augmented with
interaction terms of the form:

\begin{equation}
    Y_{st}
    = \beta_1 \text{Policy}_{st}
    + \beta_2 \big(\text{Policy}_{st} \times Z_s\big)
    + X_{st}'\Gamma
    + \alpha_s + \lambda_t + \varepsilon_{st},
\end{equation}

where $Z_s$ denotes a time-invariant state characteristic, such as political orientation, baseline
STI prevalence, or population density. The coefficient $\beta_2$ captures heterogeneous responses
to the policy shock.


% =========================================================

\section{Results}\label{sec:results}

111

\subsection{Baseline Estimates}
\subsection{Dynamic Effects (Event Study)}
\subsection{Synthetic Control Results}
\subsection{Robustness and Sensitivity Analysis}
\subsection{Heterogeneity Results}

% =========================================================

\section{Discussion}\label{sec:discussion}
111

\subsection{Interpretation and Potential Channels}

Although the empirical analysis documents a clear increase in STI outcomes following the 
post-\textit{Dobbs} abortion restrictions, the mechanisms underlying this relationship are 
potentially complex. Classical economic reasoning provides one possible pathway. By raising the 
expected cost of unintended pregnancy, restrictive abortion policy may reduce certain types of 
sexual risk-taking. This mechanism is conceptually related to the framework proposed by 
Klick and Stratmann (2003), who argue that the legalization of abortion lowered the cost of 
unprotected sexual activity and contributed to higher STI rates. A symmetric interpretation would 
suggest that restricting abortion access could reduce risk-taking and exert downward pressure on 
STI transmission.

However, several countervailing mechanisms operate in the opposite direction and may be more 
relevant in practice. Reductions in access to reproductive-health services following the 
implementation of restrictive laws can limit opportunities for STI testing, counselling, and other 
preventive services, particularly when clinics that previously provided a broad set of sexual-
health functions scale back operations. Behavioural adjustments may also shift in ways that do not 
necessarily reduce STI exposure. Some individuals may alter contraceptive choices in response to 
changes in pregnancy-related risk without making corresponding adjustments in protection against 
STIs. In addition, states enforcing restrictive abortion laws often differ in their sexual-education 
systems and broader public-health environments, which could shape patterns of sexual behaviour and 
preventive practices over longer horizons.

These channels imply that the direction of the overall effect is theoretically ambiguous. On the 
one hand, higher expected pregnancy costs may reduce certain forms of risk-taking; on the other, 
constrained access to screening, counselling, and reproductive-health infrastructure may facilitate 
greater STI transmission. The empirical results in this study provide evidence on which of these 
forces dominates in the post-\textit{Dobbs} context, but the mechanisms underlying the observed 
outcomes remain multifaceted and should be interpreted with caution.

\subsection{Relation to Existing Literature}
\subsection{Policy Implications}

\section{Conclusion}\label{sec:conclusion}
\subsection{Summary}
\subsection{Limitations}
\subsection{Directions for Future Research}

% -------------------------------------------------
%                   REFERENCES
% -------------------------------------------------
\newpage
\bibliographystyle{apalike}
\bibliography{references}

% -------------------------------------------------
%                   APPENDIX
% -------------------------------------------------
\appendix
\section{Additional Figures}
\section{Additional Tables}
\section{Robustness Checks}
\section{Data Appendix}

\end{document}