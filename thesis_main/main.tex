% !TEX root = main.tex
\documentclass{report} % or even: book | or the koma classes: scrreprt, scrbook
% or for a small thesis 'article' or the corresponding 'scrartcl'
\usepackage{microtype}
\usepackage[<encoding>]{fontenc} % probabilly: T1
\usepackage[<encoding>]{inputenc} % probabilly: utf8
%\usepackage{palatino} % just as a matter of taste
\usepackage[<your language(s)>]{babel}
\usepackage{geometry} % and then \geometry{<settings>}
\usepackage{csquotes} % probabilly with the option: autostyle=true
\usepackage{ellipsis}
\usepackage{natbib} % or biblatex
\usepackage{graphicx}
%\graphicspath{ {images/} } % or whatever your "images"-directory is
\usepackage{todonotes} % or fixme
\usepackage{fancyhdr}
\usepackage{emptypage}
\usepackage{hyperref}

...

%declaration environment
\usepackage{titling}

\makeatletter
\newif\if@decltotoc
\newcommand\declarationname{Declaration of Authorship}
\newcommand\ltx@sectionings{chapter,section,subsection,subsubsection,subparagraph}
\newcommand\decl@rationsect{chapter}
\ifdefined\chapter\else\renewcommand\decl@rationsect{section}\fi
\newenvironment{declaration}[2][\decl@rationsect]{%
  \edef\@tempa{\decl@rationsect}%
  \edef\reserved@a{#1}%
  \gdef\theplace{#2}%
  \@decltotocfalse
  \@ifundefined{#1}{\@latex@warning{#1 not defined}}{%
    \@tempswafalse%
    \@for\sec:=\ltx@sectionings\do{\ifx\sec\reserved@a\@tempswatrue\fi}%
    \if@tempswa\let\@tempa\reserved@a\else
      \@latex@warning{#1 is not a sectioning command, so I overrode it}\fi}
  \csname\@tempa\endcsname*{\declarationname}
  \if@decltotoc\addcontentsline{toc}{\@tempa}{\declarationname}\fi
}{%
  \par\vskip6em\par\noindent\theauthor\hfill\theplace,\space\thedate\par
  \global\let\declaration\gobble@env
  \global\let\enddeclaration\relax
  \global\expandafter\let\csname enddeclaration*\endcsname\relax
}
\expandafter\def\csname declaration*\endcsname{\let\@decltotocfalse\@decltotoctrue\declaration}
\expandafter\let\csname enddeclaration*\endcsname\enddeclaration
\def\gobble@env{\@ifnextchar[{\@gobble@env}{\@gobble@env[]}}
\def\@gobble@env[#1]{\@bsphack\@@gobble@env}
\def\@@gobble@env#1\end{\@esphack\end}
\makeatother

% ========== Meta Information ==========
\title{The Impact of Abortion Accessibility on Sexually Transmitted Infections:
Evidence from the U.S. Dobbs Decision}
\author{Aaron Liu}
\date{Master's Thesis \\ Graduate School of Economics \\ The University of Tokyo \\ November 2025}

% ========== Section formatting ==========
\titleformat{\chapter}{\Large\bfseries}{\thechapter.}{1em}{}
\titleformat{\section}{\large\bfseries}{\thesection}{1em}{}

% ========== Document ==========
\begin{document}

\graphicspath{{figures/}}
\maketitle
\tableofcontents
\listoftables
\listoffigures

% ==========================================================
\chapter{Introduction}
\section{Background and Motivation}
Recent developments in abortion law reforms, particularly following the \textit{Dobbs v. Jackson Women’s Health Organization} decision, have significantly altered reproductive healthcare accessibility across U.S. states. 
This thesis investigates whether and how these policy changes have affected the transmission of sexually transmitted infections (STIs). 

\section{Research Question and Contribution}
The central question is whether restrictive abortion policies increase STI incidence by reducing access to sexual health services or altering risk behaviors. 
This study contributes by combining a causal identification strategy (Difference-in-Differences) with modern machine learning techniques (Double Machine Learning), complemented by heterogeneity and behavioral analyses.

\section{Overview of Methodology}
This study integrates four empirical approaches: 
(1) a state-level two-way fixed-effects DID model as the main identification, 
(2) a DML framework for robustness under flexible controls, 
(3) heterogeneity and synthetic control analyses to explore cross-state differences, 
and (4) Google Trends analysis as behavioral evidence.

% ==========================================================
\chapter{Institutional Background and Literature Review}
\section{Abortion Law Reforms in the United States}
Provide a timeline of major reforms (e.g., trigger laws, gestational limits, enforcement mechanisms).

\section{Public Health and Reproductive Policy Linkages}
Summarize literature linking reproductive rights, healthcare access, and sexual behavior.

\section{Related Empirical Studies}
Discuss key works using DID, synthetic control, or DML in policy evaluation.

\section{Positioning of This Study}
Clarify how your work differs: cross-state STI focus, modern methods, behavioral mechanism.

% ==========================================================
\chapter{Data and Empirical Strategy}
\section{Data Sources}
\begin{itemize}
    \item \textbf{STI Data111:} CDC AtlasPlus – Syphilis, Gonorrhea, and Chlamydia rates (2010–2023)
    \item \textbf{Socioeconomic Controls:} Census ACS, SAIPE, BLS, BEA datasets
    \item \textbf{Policy Variables:} Abortion law timelines, cohort identifiers
    \item \textbf{Behavioral Data:} Google Trends indices on sexual health keywords
\end{itemize}

\section{Baseline Identification: Difference-in-Differences}
\begin{equation}
Y_{st} = \alpha + \beta (Post_t \times Treated_s) + X_{st}'\gamma + \mu_s + \lambda_t + \varepsilon_{st}
\end{equation}
where $Y_{st}$ denotes STI incidence in state $s$ and year $t$, and $\beta$ represents the average treatment effect on the treated (ATT).

\section{Double Machine Learning Framework}
\begin{equation}
Y = \theta D + g(X) + \varepsilon, \quad D = m(X) + \nu
\end{equation}
The estimator $\hat{\theta}$ captures the average treatment effect (ATE), controlling flexibly for high-dimensional confounders via Random Forest learners.

\section{Heterogeneity Analysis}
Subgroup and interaction analysis across socioeconomic environments (poverty, education, insurance).

\section{Synthetic Control Design}
Construct state-specific counterfactuals (e.g., Texas, Louisiana) to visualize dynamic policy effects.

% ==========================================================
\chapter{Empirical Results}
\section{Baseline DID Results}
Present main regression tables for Syphilis, Gonorrhea, Chlamydia, and composite indices.

\section{Robustness: Double Machine Learning Estimates}
Report ATEs estimated via DML-PLR; compare direction and magnitude to DID results.

\section{Heterogeneity Across Socioeconomic Environments}
Show results for high/low poverty, education, and insurance groups.

\section{Synthetic Control Case Studies}
Include gap plots and placebo tests for representative states.

% ==========================================================
\chapter{Mechanism Evidence: Behavioral Responses}
\section{Google Trends Analysis}
Describe keyword selection (e.g., ``condom use'', ``STD testing'', ``birth control clinic'') and construction of state-level indices.

\section{Empirical Strategy}
Use event-study or DID-style comparison around reform years to test behavioral responses.

\section{Findings and Interpretation}
Discuss how reductions in preventive-search behavior correspond to increases in STI rates.

% ==========================================================
\chapter{Conclusion and Discussion}
\section{Summary of Findings}
Recap the main empirical findings and their implications.

\section{Policy Implications}
Highlight implications for public health and reproductive policy coordination.

\section{Limitations and Future Research}
Discuss data constraints, unobserved spillovers, and potential extensions (e.g., county-level analysis, Title X clinic mechanisms).

% ==========================================================
\begin{appendices}
\chapter{Appendix: Data Construction Details}
Include variable definitions, FIPS mappings, and API sources.

\chapter{Appendix: Additional Tables and Figures}
Event-study coefficients, placebo results, robustness checks.

\chapter{Appendix: Code Summary}
List major R scripts for DID, DML, heterogeneity, SCM, and Google Trends.
\end{appendices}

% ==========================================================
\bibliographystyle{apalike}
\bibliography{references}

\end{document}