\documentclass{article}
\usepackage{graphicx} % Required for inserting images
\usepackage{amsmath}
\usepackage{amssymb}
\usepackage{hyperref}
\usepackage{booktabs}
\usepackage{threeparttable}
\usepackage{longtable}
\usepackage{caption}
\usepackage{subcaption}



% =========================================================

\begin{document}

\newpage
\tableofcontents
\newpage

\listoffigures

\listoftables
\newpage

\pagenumbering{arabic}


\section{Introduction}

\subsection{Research Question}

This study examines whether the abortion restrictions enacted in the aftermath of the
\textit{Dobbs v.~Jackson Women’s Health Organization} ruling produced measurable effects on
sexually transmitted infection (STI) outcomes in the United States. The \textit{Dobbs} decision
fundamentally altered the institutional environment surrounding reproductive health by removing
federal constitutional protection for abortion and returning regulatory authority to individual
states. Several states responded by enforcing restrictive laws almost immediately, while others
maintained the status quo. This abrupt divergence provides a policy intervention that is plausibly
exogenous with respect to underlying STI trends.

The central question is whether these restrictive abortion policies contributed to changes in the
incidence of major reportable STIs—including syphilis, gonorrhea, and chlamydia—during the two
years following the ruling. STIs have exhibited sustained growth in the United States over the
past decade, and recent policy debates have highlighted the importance of reproductive-health
infrastructure and sexual health services in shaping disease transmission. Despite these broader
concerns, empirical evidence linking abortion policy to STI trends remains limited, and the
potential spillovers of reproductive policy into other dimensions of public health are still not
well understood.

This study seeks to provide a systematic assessment of whether the policy shift in 2022 induced
observable departures from the counterfactual STI trajectories that states would have followed in
the absence of the ruling. The analysis is grounded in state-year panel data spanning more than a
decade before and after the decision, allowing for the estimation of both average treatment
effects and dynamic changes over time. By bringing together policy variation, detailed public
health data, and modern causal inference methods, the study aims to clarify whether the post-
\textit{Dobbs} abortion restrictions have had unintended consequences for sexual health outcomes.


\subsection{Motivation}

Reproductive-health policies often influence a wide range of behavioural and institutional
responses beyond fertility outcomes. Following the \textit{Dobbs} decision, substantial public
debate has focused on access to reproductive care, the reconfiguration of healthcare systems, and
potential spillovers to related health behaviours. At the same time, STI rates in the United
States have exhibited persistent growth in the past decade, with notable heterogeneity across
states. Despite these developments, empirical evidence on the causal link between post-\textit{Dobbs}
abortion restrictions and STI outcomes remains limited. The motivation of this study is to fill
this gap by providing credible causal estimates based on long-term state-level panel data.

\subsection{Contribution}

This study contributes to the literature in several ways. First, it provides new causal evidence on 
how abortion restrictions influence sexual-health outcomes, an area that has received limited 
empirical attention. While existing work has examined the effects of reproductive-health policy on 
fertility, abortion incidence, and related services, the potential spillovers onto STI outcomes have 
not been systematically analysed. Second, the study constructs a state-year panel from 2010 to 2023 
that combines surveillance data, demographic indicators, and socioeconomic controls. This dataset 
supports both cross-state comparisons and dynamic analysis over an extended period. Third, the 
empirical design integrates Difference-in-Differences estimation, event-study specifications, and 
synthetic control methods. This combination provides a transparent framework for assessing the 
robustness of the findings and evaluating the credibility of identification. By triangulating 
evidence across complementary approaches, the analysis aims to produce a coherent assessment of the 
policy’s impact.

\subsection{Summary of Findings}

The results indicate that ...


\subsection{Roadmap}

The remainder of the thesis proceeds as follows.
Section~\ref{sec:literature} reviews the relevant literature on abortion policy, sexually transmitted
infections, and public health outcomes.
Section~\ref{sec:institution} describes the institutional and policy background relevant to the
\textit{Dobbs} ruling and the variation in state-level abortion restrictions.
Section~\ref{sec:data} presents the data sources, variable construction, and descriptive
statistics.
Section~\ref{sec:strategy} outlines the empirical strategy, including the baseline Difference-in-
Differences model, event-study framework, and robustness checks.
Section~\ref{sec:results} reports the main results.
Section~\ref{sec:discussion} discusses the broader interpretation of the findings and situates
them within the existing literature.
Section~\ref{sec:conclusion} concludes.

% =========================================================
\section{Literature Review}\label{sec:literature}

% =========================================================

\section{Institutional Background}
\label{sec:institution}

This section situates the empirical analysis within the institutional structure
shaped by the post-\textit{Dobbs} abortion policy environment. The removal of
federal protection altered the relative costs and constraints associated with
fertility control, thereby modifying the incentives governing sexual behaviour,
healthcare utilization, and cross-state mobility. Because sexually transmitted
infections (STIs) respond to both behavioural choices and access to medical
services, the sharp policy discontinuities generated by \textit{Dobbs}
constitute a relevant institutional shock with measurable implications for
public health outcomes. Figures~\ref{fig:policy_map} and
\ref{fig:sti_map} summarize these institutional features by depicting the
distribution of state abortion laws and baseline STI incidence.

\subsection{Abortion Policy After \textit{Dobbs}}

The Supreme Court’s 2022 decision in \textit{Dobbs v. Jackson Women’s Health
Organization} rescinded the federal constitutional right to abortion and
returned full regulatory authority to the states. This reassignment produced
large cross-state variation in the effective price of terminating a pregnancy,
both in monetary and non-monetary terms (e.g., travel costs, waiting time,
legal risk, and uncertainty). From an economic perspective, these changes alter
the feasible set of fertility-control options available to individuals and
reshape the intertemporal trade-offs between sexual activity, contraceptive
behaviour, and reproductive planning.

States responded heterogeneously. Some activated pre-existing bans, sharply
raising the cost of abortion, while others maintained permissive regimes. The
resulting divergence generates a clear treated--control structure in which
otherwise comparable states face different institutional constraints on
reproductive choice. Because these changes were abrupt and externally imposed,
they provide a quasi-experimental setting for identifying behavioural and
health-related responses.

\subsection{Trigger Laws and State-Level Variation}

Prior to \textit{Dobbs}, several states had enacted ``trigger laws'' designed to
take effect immediately upon the removal of federal protections. These statutes
implemented near-total bans or stringent gestational limits, substantially
raising the marginal cost of obtaining an abortion within the state. From a
behavioural standpoint, such policies modify the relative prices of precaution,
risk-taking, and healthcare substitution. Individuals may respond by adjusting
sexual behaviour, shifting toward less protected intercourse, or reallocating
healthcare demand across states.

Other states adopted new restrictions in the months following the ruling, while
a separate group preserved broad legal access. This staggered implementation
introduces cross-sectional and temporal variation that is economically
meaningful: variation in policy intensity interacts with mobility frictions,
local clinic capacity, and the substitutability of out-of-state services. As
illustrated in Figure~\ref{fig:policy_map}, these differences form the core
source of identifying variation used in the empirical strategy.

\subsection{STI Trends and Public Health Framework}

STIs are tracked through a federal--state surveillance framework that records
diagnoses, demographic characteristics, and case counts at the state level.
From an economic perspective, STI incidence reflects both the demand side
(behavioural risk-taking, screening decisions, healthcare access) and the supply
side (availability of sexual health services, testing capacity, and reporting
infrastructure). These determinants vary across states because of differences in
demographics, income distribution, insurance coverage, and the density of
clinics providing reproductive and preventive care.



\begin{figure}
    \centering
    \includegraphics[width=1\textwidth]{figures/std_trends_barplot.png}
    \caption{State-Year Trends in STI Incidence (2010–2023)}
\end{figure}

\begin{figure}[htbp]
\centering

\begin{subfigure}{0.8\textwidth}
    \centering
    \includegraphics[width=\textwidth]{figures/ch_index_diff.png}
    \caption{Chlamydia}
\end{subfigure}

\vspace{6pt}

\begin{subfigure}{0.8\textwidth}
    \centering
    \includegraphics[width=\textwidth]{figures/go_index_diff.png}
    \caption{Gonorrhea}
\end{subfigure}

\vspace{6pt}

\begin{subfigure}{0.8\textwidth}
    \centering
    \includegraphics[width=\textwidth]{figures/sy_index_diff.png}
    \caption{Syphilis}
\end{subfigure}

\caption{Changes in STD Incidence Between 2021 and 2023}
\label{fig:std_diff_all}
\end{figure}

% =========================================================
\section{Data}
\label{sec:data}

This section describes the construction of the state--year panel covering the
period 2010--2023. The dataset combines information on sexually transmitted
infection (STI) outcomes, abortion policy environments, and a set of
socioeconomic and demographic indicators. The integration of these sources
produces a balanced panel with consistent measurement across states and time,
providing the empirical foundation for identifying the effects of
post-\textit{Dobbs} abortion restrictions.

\subsection{Data Sources}

The primary outcome measures come from the Centers for Disease Control and
Prevention (CDC) annual surveillance reports, which provide state-level
incidence rates for major notifiable STIs. Information on abortion policies is
assembled from statutory records, legal documents concerning the activation of
trigger laws, and state-level implementation timelines following the
\textit{Dobbs} decision. Demographic and socioeconomic variables are drawn from
the American Community Survey (ACS), the Bureau of Labor Statistics (BLS), and
the Small Area Income and Poverty Estimates (SAIPE) program. Additional
harmonized population counts and age distributions are obtained from IPUMS
NHGIS. Together, these sources supply a consistent and well-documented set of
variables for empirical analysis.

\subsection{Outcome Variables}

The analysis focuses on three commonly reported infections: syphilis,
gonorrhea, and chlamydia. For each infection, the CDC reports confirmed case
counts and incidence rates per 100,000 population. Incidence rates are used to
standardize across states with different population sizes. Some specifications
employ logarithmic transformations of incidence measures to reduce the influence
of skewness and facilitate interpretation in proportional terms.

\subsection{Treatment Variable}

The treatment variable indicates whether a state enforced a restrictive abortion
statute in a given year. A state enters the treated group in the first year in
which its trigger law or other prohibition became legally operative.
Non-restrictive states form the control group throughout the sample window. The
coding captures the effective timing of legal implementation rather than
legislative announcements and excludes states whose adoption occurred too late
to furnish adequate pre-treatment observations. This definition corresponds
directly to the quasi-experimental variation underlying the empirical design.

\subsection{Socioeconomic and Demographic Controls}

To account for time-varying factors that may influence STI transmission or
correlate with policy adoption, the regression models include controls for
population age structure, gender composition, racial composition, educational
attainment, unemployment, and poverty. These variables are harmonized to the
state--year level using ACS, SAIPE, and BLS releases. Their inclusion serves to
capture changes in demographic risk profiles and economic conditions that could
affect STI outcomes independently of abortion policy.

\subsection{Sample Construction and Cleaning}

All data sources are merged by standardized FIPS codes. Observations with
incomplete or implausible values are removed following consistent criteria.
Missing values for controls are handled through established practices, including
interpolation or exclusion depending on the variable’s stability and relevance.
States with late adoption of abortion restrictions are excluded from the main
treated--control contrast to ensure sufficient pre-treatment periods for
Difference-in-Differences estimation. The final dataset is a balanced panel of
states observed continuously from 2010 to 2023.

\subsection{Descriptive Statistics}

Descriptive statistics document substantial cross-state heterogeneity in STI
incidence, demographic structure, and socioeconomic conditions. The geographic
distribution of baseline STI burdens and statutory abortion regimes, illustrated
in Section~\ref{sec:institution}, highlights the coexistence of persistent
structural differences across states. These patterns justify the use of
fixed-effects estimators that absorb time-invariant heterogeneity and motivate
the examination of how post-\textit{Dobbs} restrictions shifted STI trajectories
relative to their counterfactual evolution.

\input{tables/desc_stats.tex}
% =========================================================

\section{Empirical Strategy}\label{sec:strategy}

This section outlines the empirical approach used to estimate the impact of post-\textit{Dobbs}
abortion restrictions on sexually transmitted infection outcomes. The design takes advantage of
a sharp institutional shift produced by the 2022 Supreme Court decision, which removed federal
constitutional protection for abortion and led to immediate policy changes in a subset of states.
The staggered response across states creates a treated–control structure in which some states
implemented restrictive laws while others maintained existing access. This variation, combined
with a long state-year panel from 2010 to 2023, provides a suitable environment for identifying
causal effects using Difference-in-Differences methods, dynamic event-study estimators, and
synthetic control analysis. These approaches enable the estimation of both average and evolving
policy effects while offering multiple avenues for assessing identification and robustness.

\subsection{Baseline Difference-in-Differences Model}

The analysis begins with a two-way fixed-effects Difference-in-Differences specification. Let
$Y_{st}$ denote the STI outcome of interest for state $s$ in year $t$. The treatment indicator
$\text{Policy}_{st}$ equals one if a state enforces a restrictive abortion statute in that year
and zero otherwise. The baseline model is given by

\begin{equation}
    Y_{st} = \beta\,\text{Policy}_{st} + X_{st}'\Gamma + \alpha_s + \lambda_t + \varepsilon_{st}.
\end{equation}

The vector $X_{st}$ consists of time-varying covariates that capture demographic and
socioeconomic conditions that may influence STI dynamics and correlate with policy adoption.
State fixed effects $\alpha_s$ remove persistent differences across states, such as long-standing
cultural, institutional, or healthcare infrastructure characteristics. Year fixed effects
$\lambda_t$ absorb national developments that influence all states simultaneously, including
changes in reporting practices or nationwide health shocks. Standard errors are clustered at the
state level to account for serial correlation.

Under the identifying assumptions discussed in the next subsection, the coefficient $\beta$
captures the average treatment effect of abortion restrictions on STI outcomes. This estimate
reflects how the treated states’ trajectories diverged from their counterfactual paths represented
by control states. The specification serves as the foundation for subsequent analyses and provides
a benchmark against which more flexible approaches can be compared.

\subsection{Identification Assumptions}

The interpretation of the Difference-in-Differences coefficient as a causal effect depends on the plausibility of several identification assumptions. A first requirement is that treated and untreated states would have followed parallel trends in STI outcomes in the absence of policy changes. While the counterfactual cannot be observed directly, the dynamic event-study specifications presented below allow for an empirical assessment of pre-treatment dynamics. Prior to the policy shift, STI trends were broadly stable across states, and there is no indication of systematic divergence between future adopters of restrictive laws and states that maintained pre-\textit{Dobbs} access.

A second requirement is the absence of anticipatory behaviour. Individuals, clinics, and health agencies should not adjust their actions before the formal implementation of abortion restrictions. The leak of the draft \textit{Dobbs} opinion raises the possibility of early behavioural responses, particularly in the months preceding the official ruling. In the empirical analysis, pre-treatment coefficients provide evidence on whether such anticipatory adjustments meaningfully affect the identification of the treatment effect.

A third and conceptually distinct assumption concerns the no-interference component of the Stable Unit Treatment Value Assumption (SUTVA). This requirement implies that the potential outcomes of a state must depend solely on its own treatment status and not on the treatment assignments of other states. In the present setting, STI outcomes in a non-restrictive state such as Illinois should be unaffected by restrictive statutes adopted in neighbouring states such as Missouri. In practice, however, state borders are porous and some degree of spillover is possible through channels such as cross-state mobility for abortion services, compositional changes in patient populations, or nationwide information and behavioural responses following the \textit{Dobbs} decision. If such spillovers occur, they attenuate the contrast between treated and control states, implying that the estimated policy effect is biased toward zero and may be interpreted as a conservative lower bound on the true causal impact.

Together, these assumptions form the conceptual basis for the empirical strategy. Their plausibility is reinforced by the event-study evidence, which demonstrates stable pre-treatment dynamics and a clear divergence emerging only after the policy shock.

\subsection{Event Study Specification}\label{sec:eventstudy}

To assess pre-treatment dynamics and document the evolution of treatment effects over time, the
DID model is expanded into an event-study specification. The model replaces the single treatment
indicator with a set of relative-time indicators that capture periods before and after the policy
change:

\begin{equation}
    Y_{st} = \alpha_s + \lambda_t +
    \sum_{k=-M}^{K} \beta_k D_{st}^{(k)} + X_{st}'\Gamma + \varepsilon_{st},
\end{equation}

where $D_{st}^{(k)}$ equals one if year $t$ is $k$ years relative to the implementation year,
with $k=0$ corresponding to 2022. The omitted category is $k=-1$, allowing the remaining
coefficients to be interpreted relative to the year immediately preceding the policy shift.

The event-study serves two purposes. First, the coefficients associated with pre-treatment periods
provide evidence on the validity of the parallel-trends and no-anticipation assumptions. A lack of
significant movement in these coefficients strengthens the credibility of the causal
interpretation. Second, the post-treatment coefficients trace the dynamic path of the policy’s
effects, distinguishing immediate responses from more gradual adjustments. The timing of the
\textit{Dobbs} ruling, concentrated in a single year with a substantial never-treated group,
circumvents common concerns regarding staggered adoption and heterogeneous treatment timing,
making the event-study estimator well suited for this setting.

\subsection{Synthetic Control Design}

To complement the DID and event-study results, synthetic control methods are employed for selected
treated states. For each treated state examined, a synthetic counterpart is constructed as a
convex combination of untreated states chosen to closely reproduce the treated state’s
pre-treatment STI trajectory. The weights are selected to minimize discrepancies in outcomes and
predictor variables during the pre-2022 period.

A credible synthetic control requires a strong pre-treatment fit. The root mean squared prediction
error (RMSPE) is used to assess fit quality, with a lower RMSPE indicating a more reliable
counterfactual. Following the construction of the synthetic unit, the analysis compares the paths
of the treated state and its synthetic counterpart after 2022. A widening post-treatment gap is
interpreted as the effect of the policy. To evaluate statistical significance, placebo procedures
are implemented by repeating the synthetic control construction for each state in the donor pool.
The distribution of these placebo gaps provides a benchmark for assessing the magnitude of the
observed effects.

\subsection{Robustness Checks}

Several robustness checks assess whether the findings depend on particular modelling choices or
data features. A first exercise assigns the treatment to a pre-\textit{Dobbs} period in which no
policy change occurred. If the empirical strategy were sensitive to unrelated fluctuations, this
placebo treatment would generate effects similar to the baseline estimates. A lack of significant
placebo effects therefore reinforces the credibility of the main results.

A second robustness exercise arises within the synthetic control framework. By applying the same
construction to each state in the donor pool, the analysis generates a distribution of placebo
paths for untreated states. Comparing the magnitude of the actual post-treatment gap to this
distribution provides a non-parametric test of the policy’s impact.

A third robustness exercise addresses the exceptional nature of the COVID-19 period. The years
2020 and 2021 were marked by disruptions in healthcare utilisation, reporting, and behavioural
patterns relevant to STI transmission. Including these years as part of the pre-treatment period
could distort the counterfactual trend. The main specification therefore omits these years from
the estimation sample, and additional analyses confirm that the key findings are not driven by the
treatment of the pandemic period.

Across all robustness exercises, the estimated effects remain consistent in sign and magnitude,
providing further support for the interpretation that the post-\textit{Dobbs} increase in STI
outcomes reflects the causal impact of abortion restrictions rather than artefacts of model
specification or data irregularities.

\subsection{Heterogeneity Analysis}

To examine whether the policy effect varies across states, the baseline model is augmented with
interaction terms of the form:

\begin{equation}
    Y_{st}
    = \beta_1 \text{Policy}_{st}
    + \beta_2 \big(\text{Policy}_{st} \times Z_s\big)
    + X_{st}'\Gamma
    + \alpha_s + \lambda_t + \varepsilon_{st},
\end{equation}

where $Z_s$ denotes a time-invariant state characteristic, such as political orientation, baseline
STI prevalence, or population density. The coefficient $\beta_2$ captures heterogeneous responses
to the policy shock.


% =========================================================

\section{Results}\label{sec:results}

\subsection{Event Study Analysis}

\begin{figure}[htbp]
\centering

\begin{subfigure}{0.48\textwidth}
    \centering
    \includegraphics[width=\textwidth]{figures/event_study_plot_Chlamydia.png}
    \caption{Chlamydia}
\end{subfigure}
\hfill
\begin{subfigure}{0.48\textwidth}
    \centering
    \includegraphics[width=\textwidth]{figures/event_study_plot_Syphilis.png}
    \caption{Syphilis}
\end{subfigure}

\vspace{8pt}

\begin{subfigure}{0.48\textwidth}
    \centering
    \includegraphics[width=\textwidth]{figures/event_study_plot_Gonorrhea.png}
    \caption{Gonorrhea}
\end{subfigure}
\hfill
\begin{subfigure}{0.48\textwidth}
    \centering
    \includegraphics[width=\textwidth]{figures/event_study_plot_STDs.png}
    \caption{Composite Index}
\end{subfigure}

\caption{Event Study Estimates for Four STD Outcomes}
\label{fig:event_study_all}
\end{figure}

To examine the validity of the parallel trends assumption and to explore the
dynamic response of STD incidence around the implementation of abortion
restrictions, I estimate an event--study specification that replaces the single
post–treatment indicator with time–relative dummies. The coefficients trace the
difference between treated and control states in each period relative to the
year immediately preceding the policy change.

The pre–treatment coefficients remain close to zero across all diseases and do
not display systematic patterns. Their confidence intervals generally include
zero, and there is no evidence of differential pre–policy movements between
treated and control states. These findings support the parallel trends
assumption underlying the DID research design.

Following the adoption of abortion restrictions in 2022, the event–study plots
exhibit heterogeneous adjustment paths across disease categories. For
chlamydia, the post–treatment coefficients rise persistently and become
statistically distinguishable from zero within the first two years after the
policy change. The pattern indicates a gradual and monotonic increase in
reported chlamydia cases in treated states relative to control states.
For syphilis, the post–policy coefficients are positive as well, although their
precision is lower and the confidence intervals are wider. The magnitude of the
estimated effects is smaller compared to chlamydia, and significance is reached
in fewer post–periods. In contrast, the estimated effects for gonorrhea remain
centered around zero throughout the post–treatment horizon, and none of the
coefficients is statistically significant. The composite STD index shows a
modest positive shift after 2022, consistent with the pattern observed in
chlamydia, but with smaller magnitude.

Taken together, the event–study evidence suggests that treated and control
states follow comparable trends before the policy change, supporting the use of
a DID estimator. The dynamic post–treatment responses further indicate that the
impact of abortion restrictions is not uniform across diseases: chlamydia shows
a clear upward adjustment, syphilis displays weaker and less precisely measured
effects, and gonorrhea exhibits no detectable change. These dynamic patterns are
consistent with the heterogeneity found in the main DID estimates presented in
the next section.




\subsection{Difference-in-Differences Estimates}

\input{tables/did_chlamydia.tex}
\input{tables/did_syphilis.tex}
\input{tables/did_gonorrhea.tex}
\input{tables/did_stds_composite.tex}

Tables~\ref{tab:did_chlamydia} to \ref{tab:did_composite} present the
Difference-in-Differences (DID) estimates across four model specifications:
a baseline model without controls, models including demographic controls 
and the full set of socioeconomic controls, and a specification that additionally
incorporates state-specific linear trends. The coefficient on 
\textit{Post $\times$ Treated} captures the average policy effect.

For chlamydia, the estimated coefficients are consistently positive and
statistically significant across all specifications. The magnitude of the effect
remains stable as additional controls are introduced, and the positive and
significant impact persists even after allowing for state-specific linear trends.
These results indicate a systematic post-2022 increase in reported chlamydia cases
in states enforcing abortion restrictions, relative to control states.

For syphilis, the treatment effect is positive in all models.
It reaches statistical significance in the baseline specification and
remains directionally stable when controls are added.  
The model including state-specific trends again yields a marginally significant
estimate. Although the overall pattern points to an increase in syphilis cases
following the policy change, the level of statistical significance is weaker
than in the chlamydia results.

In contrast, the estimates for gonorrhea are small in magnitude and never reach
statistical significance. The inclusion of demographic or full controls does not
alter this conclusion. These findings suggest that the policy does not generate a
detectable average effect on gonorrhea incidence during the sample period.

Finally, the composite STD index shows a modest positive impact.
The baseline specification and the model with state-specific trends yield
statistically significant coefficients, whereas the intermediate specifications
produce estimates of similar sign but lower precision. The composite results align
with the disease-specific patterns, indicating a mild increase in overall STD
incidence following the implementation of abortion restrictions.

Taken together, the results reveal heterogeneous patterns across diseases:
chlamydia exhibits the most robust and consistently significant increase;  
syphilis shows a positive but less precisely estimated effect;  
gonorrhea displays no discernible response;  
and the composite index records a small overall rise.  
The stability of the estimates across specifications indicates that the main
findings are not sensitive to model choice.




\subsection{Synthetic Control Results}

\begin{figure}
    \centering
    \includegraphics[width=\textwidth]{figures/synth_average_gap_plot.png}
    \caption{Synthetic Control Results}
    \label{fig:sc_average_gap}
\end{figure}

Figure~\ref{fig:sc_average_gap} displays the synthetic control comparison for the
treated states. The pre–treatment trajectories of the treated and synthetic
series move closely together, indicating a satisfactory pre–policy fit and
suggesting that the donor pool provides a credible counterfactual benchmark.
After the implementation of abortion restrictions in 2022, the treated outcome
diverges sharply from its synthetic counterpart. While the synthetic control
continues along a relatively stable path, the treated trajectory rises
substantially, generating a widening positive gap in the post–treatment period.
The timing and magnitude of this divergence align with the DID and event–study
results and provide complementary evidence of an increase in STD incidence
following the policy change.




\subsection{Robustness and Sensitivity Analysis}

\begin{figure}
    \centering
    \includegraphics[width=\textwidth]{figures/placebo_event_study_sy_2016.png}
    \caption{Placebo Test Results for Syphilis (Placebo Treatment in 2016)}
    \label{fig:placebo_sy}
\end{figure}

Figure~\ref{fig:placebo_sy} displays a placebo test assigning a fictitious treatment year to 2016 finds no discernible changes around the false treatment date. The pre–placebo estimates remain near zero, and no systematic dynamic response is observed. This indicates that the empirical strategy does not generate spurious effects in periods with no policy changes.

\subsection{Heterogeneity Results}

% =========================================================

\section{Discussion}\label{sec:discussion}

\subsection{Interpretation and Potential Channels}

Although the empirical analysis documents a clear increase in STI outcomes following the 
post-\textit{Dobbs} abortion restrictions, the mechanisms underlying this relationship are 
potentially complex. Classical economic reasoning provides one possible pathway. By raising the 
expected cost of unintended pregnancy, restrictive abortion policy may reduce certain types of 
sexual risk-taking. This mechanism is conceptually related to the framework proposed by 
Klick and Stratmann (2003), who argue that the legalization of abortion lowered the cost of 
unprotected sexual activity and contributed to higher STI rates. A symmetric interpretation would 
suggest that restricting abortion access could reduce risk-taking and exert downward pressure on 
STI transmission.

However, several countervailing mechanisms operate in the opposite direction and may be more 
relevant in practice. Reductions in access to reproductive-health services following the 
implementation of restrictive laws can limit opportunities for STI testing, counselling, and other 
preventive services, particularly when clinics that previously provided a broad set of sexual-
health functions scale back operations. Behavioural adjustments may also shift in ways that do not 
necessarily reduce STI exposure. Some individuals may alter contraceptive choices in response to 
changes in pregnancy-related risk without making corresponding adjustments in protection against 
STIs. In addition, states enforcing restrictive abortion laws often differ in their sexual-education 
systems and broader public-health environments, which could shape patterns of sexual behaviour and 
preventive practices over longer horizons.

These channels imply that the direction of the overall effect is theoretically ambiguous. On the 
one hand, higher expected pregnancy costs may reduce certain forms of risk-taking; on the other, 
constrained access to screening, counselling, and reproductive-health infrastructure may facilitate 
greater STI transmission. The empirical results in this study provide evidence on which of these 
forces dominates in the post-\textit{Dobbs} context, but the mechanisms underlying the observed 
outcomes remain multifaceted and should be interpreted with caution.

\subsection{Relation to Existing Literature}
\subsection{Policy Implications}

\section{Conclusion}\label{sec:conclusion}
\subsection{Summary}
\section{Limitations}

This study has several limitations that should be considered when interpreting the results. The most fundamental constraint arises from the limited availability of post-treatment data. Because the CDC is migrating the National Notifiable Diseases Surveillance System (NNDSS), state-level STD statistics for 2024 and subsequent years will not be released until 2026, leaving only two observed post-Dobbs years (2022--2023). This short post-policy window restricts the ability to trace dynamic treatment effects, assess medium- to long-run adjustments, and conduct robustness checks that depend on extended post-treatment horizons. In addition, the STD surveillance system is subject to reporting delays, heterogeneous diagnostic capacity, and differing public health infrastructure across states. Although fixed effects mitigate time-invariant heterogeneity, residual noise may persist, particularly given fluctuations in reporting intensity surrounding the COVID-19 period. Uncertainty in policy timing further introduces potential attenuation: trigger laws and gestational limits sometimes took effect gradually due to litigation, injunctions, or administrative delays, making implementation imperfectly discrete. The use of annual rather than monthly data also limits the granularity with which short-term behavioral responses can be captured, potentially obscuring transitional dynamics. Finally, the reduced-form approach estimates net policy effects but does not directly observe mechanism variables such as sexual behavior, contraception use, or clinic availability, and the institutional environment of the United States may limit the external validity of the findings.

\subsection{Directions for Future Research}

% -------------------------------------------------
%                   REFERENCES
% -------------------------------------------------
\newpage
\bibliographystyle{apalike}
\bibliography{references}

% -------------------------------------------------
%                   APPENDIX
% -------------------------------------------------
\appendix
\section{Additional Figures}
\section{Additional Tables}
\section{Robustness Checks}
\section{Data Appendix}

\end{document}